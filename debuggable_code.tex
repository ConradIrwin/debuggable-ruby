\documentclass[14pt]{beamer}

\input{include/pygments}
\input{include/itemspace}
\input{include/no-navigation}

% \usepackage{beamerthemesplit} // Activate for custom appearance
\setbeamertemplate{navigation symbols}{}%remove navigation symbols

\usepackage{ulem}

\newlength{\wideitemsep}
\setlength{\wideitemsep}{\itemsep}
\addtolength{\wideitemsep}{10pt}
\let\olditem\item
\renewcommand{\item}{\setlength{\itemsep}{\wideitemsep}\olditem}

\begin{document}

\title{Debuggable Ruby}
\author{@ConradIrwin}
\date{\today}
\frame{\titlepage}

{
\usebackgroundtemplate{\includegraphics[height=\paperheight]{img/sad-panda.jpg}}
\begin{frame}[plain]
\end{frame}
}

\frame
{
  \begin{exampleblock}{}
    {\large ``Beware of bugs in the above code; I have only proved it correct, not tried it.''}
    \vskip5mm
    \hspace*\fill{\small--- Donald Knuth}
  \end{exampleblock}
}

\frame
{
  \frametitle{Dealing with bugs}

  \begin{description}
  \item[Java]<1-> Let's make interfaces simple enough that bugs don't happen.
  \item[Haskell]<2-> Let's make the type-system powerful enough to catch all bugs.
  \item[Ruby]<3-> Bugs happen anyway, let's make it really easy to debug.
  \end{description}
}

\frame
{
  \frametitle{Debugging process}

  \begin{itemize}
  \item Find something that's broken.
  \item Narrow it down until you know why it's broken.
  \item Fix the underlying cause.
  \item Check that it now works.
  \end{itemize}
}

\newcommand{\keyword}[1]{{\color{yellow!50!black}#1}}
\newcommand{\quoted}[1]{{\color{green!60!black}#1}}

\frame
{
  \frametitle{How people debug}

  \setbeamercovered{transparent}
  \begin{itemize}
  \item<1-3>
        \only<1>{\texttt{\keyword{puts} foo.inspect }}
        \only<2->{\texttt{\keyword{p} foo}}
  \item<1-3> \texttt{\keyword{raise} \quoted{"zomg wtf bbq"} \textbf{if} foo? }
  \item<1-3> Run the code in \texttt{rails c}
  \item<1-3> Read the code\ldots
  \item<3-4> \texttt{{\color{yellow!60!black} binding}.pry}
  \end{itemize}
}

\frame
{
  \frametitle{Debuggable code}

  \begin{itemize}
  \item Easy to inspect objects.
  \item Easy to run short snippets.
  \item Easy to locate the problem.
  \end{itemize}

}

\frame
{
  \frametitle{\texttt{Object\#inspect} in Ruby}

  \begin{itemize}
  \item Lists all instance variables by default.
  \item Almost always good enough.
  \item Override if the default is too noisy.
  \item Override if you define \texttt{\#to\_s}.
  \end{itemize}
}

\begin{frame}[fragile]
  \input{build/code/object_inspect.rb}
  \tiny
  \input{build/code/object_inspect_output.rb}
\end{frame}

\begin{frame}[fragile]
  \input{build/code/active_record_inspect.rb}
  \input{build/code/active_record_inspect_output.rb}
\end{frame}

\begin{frame}[fragile]
  \input{build/code/object_inspect_url.rb}
  \vskip0.5in
  \input{build/code/url_inspect.rb}
\end{frame}

\begin{frame}[fragile]
  \frametitle{Getting the default \texttt{\#inspect} back}
  \small
  \input{build/code/default_inspect.rb}
\end{frame}

\newcommand{\crossout}[1]{\only<1>{#1} \only<2>{\sout{#1}}}
\frame
{
  \frametitle{Debuggable code}


  \begin{itemize}
  \item \crossout{Easy to inspect objects.}
  \item Easy to run short snippets.
  \item Easy to locate the problem.
  \end{itemize}

}

\frame
{
  \frametitle{Example from pry}

  \begin{itemize}
  \item Pry lets you edit methods: \hfill \\
        \texttt{[1] pry(main)> edit Module\#inspect}.
  \item Bug in the edit command.
  \item \texttt{edit Module.inspect} would sometimes redefine \texttt{Module\#inspect}.
  \end{itemize}

}

\frame
{
  \frametitle{Hard to debug}

  \begin{itemize}
  \item Pry::Commands::Edit ``method name''
  \item --- \alert<3->{Uses Pry::CodeObject to get a method }
  \item --- \alert<2->{Invokes MethodPatcher(pry, method) }
  \item \hskip 1em --- \alert<2->{Uses pry.edit to open code in vim }
  \item \hskip 1em --- \alert<1->{Evals the changed source code }
  \end{itemize}
}

\begin{frame}[fragile]
  \frametitle{Minimal test case\ldots}
  \input{build/code/terrible_minimal.rb}
\end{frame}

\begin{frame}[fragile]
  \frametitle{Minimal test case\ldots}
  \input{build/code/bad_minimal.rb}
\end{frame}

\begin{frame}[fragile]
  \frametitle{Minimal test case\ldots}
  \input{build/code/good_minimal.rb}
\end{frame}

\begin{frame}[fragile]
  \frametitle{Minimal test case\ldots}
  \input{build/code/best_minimal.rb}
\end{frame}

\frame
{
  \frametitle{Debuggable code}

  \begin{itemize}
  \item \sout{Easy to inspect objects.}
  \item \crossout{Easy to run short snippets.}
  \item Easy to locate the problem.
  \end{itemize}
}

\frame
{
  \frametitle{Don't rescue nil}

  \small
  \input{build/code/broken_find_command_for_help.rb}
}
\frame
{
  \frametitle{Don't rescue nil}

  \small
  \input{build/code/find_command_for_help.rb}
}

\frame
{
  \frametitle{When in doubt, \texttt{raise}}

  \small
  \input{build/code/do_delivery.rb}
}

\frame
{
  \frametitle{Preserve \texttt{\_\_FILE\_\_} and \texttt{\_\_LINE\_\_}}

  \small
  \input{build/code/define_callbacks.rb}
}

\frame
{
  \frametitle{Preserve \texttt{\_\_FILE\_\_} and \texttt{\_\_LINE\_\_}}

  \small
  \input{build/code/rackup_builder.rb}
}

\frame
{
  \frametitle{Good code is debuggable}

  \begin{itemize}
  \item Single Responsibility Principle
  \item Seperation of Concerns.
  \item KISS.
  \end{itemize}
}

\frame
{
  \frametitle{Final words}

  \begin{itemize}
  \item Programmers spend 50 -- 85\% of their time debugging.
  \item Costs the world \$312,000,000 each year.
  \item A little effort goes a long way.
  \end{itemize}

}

\frame
{
  \frametitle{Thanks}

  \center{@ConradIrwin}

}

\end{document}

% Continuous debugment, Continuous deployment.

% Good afternoon everyone!
%
% I have a secret I want to share.
%
% When I write code, it doesn't work.
%
% ----
%
% Well, it does sometimes. But pretty rarely. Let's say 95% of the time, I run
% my shiny new code, it crashes and burns.  This makes me a sad panda.
%
% But then I remember something Knuth once said. You know Knuth, he wrote "The
% Art of Computer Programming", and "TeX", and stuff.  he's pretty clever by
% all accounts.
%
% Beware of bugs in the above code. I've merely proved it correct, not tried
% it.
%
% Turns out Knuth has a secret too.
%
% When he writes code, it doesn't work.
%
% Well, I guess it does sometimes. Probably even more often than me. But even
% he runs his code to see whether it works.
%
% ----
%
% It turns out that most debugging happens during the time you're writing code.
% Obviously there are the really hard to find ones that only happen after
% you've deployed to production. But pretty much always, code breaks beacuse
% you tried to change it.
%
% In different language communities there are different attempts to solve this:
%
%   Java: Let's hide everything behind simple interfaces so it's easy to write
%   correct code.  Haskell: Let's make the type-system powerful enough to find
%   all the bugs before you do.  Ruby: Screw it, bugs happen. Let's make it
%   super-easy to debug.
%
% Let's face it, if you could have a compiler find bugs for you; wouldn't that
% be better than having to find the bugs yourself?  Surely we should all be
% using Haskell or Clojure or w/e. The problem is that though compilers are
% very good at spotting bugs, they're not very good at telling you what caused
% it.  In ruby, the language makes it really easy to find out what's going on.
% You can just add a print statement, every object has a default implementation
% of #inspect that lets you get a pretty god idea of what's going on. If you'd
% prefer, you can even add a binding.pry, which if it were written in java
% would probably be called an AbstractPrintStatementFactory. It's really there
% to let you make as many print statements as you need until you've found the
% problem.
%
% As usual, with great power comes great responsibility. Ruby is designed to
% make it easy to debug programs, but it requires buy in from the people who
% use it to make this work well.
%
% -----
%
% Luckily it turns out that making programs easy to debug mostly coincides with
% all the other things we'd want to do anyway.
%
%   Single-purpose methods
%   Use layers of abstraction
%   Good string representations of objects
%
% ----
%
% Let's deal with the last one first, because it's the easiest to explain.
%
% How do you debug?
%
% I was looking for two answers:
%  1. puts statements
%  2. binding.pry
%
% In every language ever made, all programmers have always used puts statements.
% Why?
% They show you something you can understand.
% In ruby everyone should use pry, because:
%
%  1. You won't mis-type it.
%  2. You have time to figure out what to print.
%  3. It pauses the program so you don't have to scan through the logs
%
% But, both print statements and binding.pry require your object to be easy to see.
%
% In ruby, this can be configured by overriding Object#inspect.
% Now usually you don't need to bother the builtin inspect has a few huge advantages:
%
%  It shows you everything
%  It's safe with recursive objects
%  It's in a format everyone already knows how to read
%
% But there are some times when you should override inspect:
%
%  If your object contains secrets (like passwords)
%  If your object contains massive amounts of text (like xml blobs)
%  If you override #to_s or #to_str
%

